\documentclass[aspectratio=169, 11pt]{beamer}

% ======================================================================
% THEME & STYLING
% ======================================================================
\usetheme{Madrid}
\usecolortheme{default}
\usefonttheme{professionalfonts}

% -- Couleurs --
\definecolor{soapblue}{HTML}{1565C0}
\definecolor{restgreen}{HTML}{2E7D32}
\definecolor{graphqlpurple}{HTML}{7B1FA2}
\definecolor{grpcorange}{HTML}{E65100}
\definecolor{darkbg}{HTML}{1A1A2E}
\definecolor{accentblue}{HTML}{0F4C81}

\setbeamercolor{palette primary}{bg=accentblue, fg=white}
\setbeamercolor{palette secondary}{bg=accentblue!80, fg=white}
\setbeamercolor{palette tertiary}{bg=accentblue!60, fg=white}
\setbeamercolor{structure}{fg=accentblue}
\setbeamercolor{title}{fg=white}
\setbeamercolor{frametitle}{fg=accentblue, bg=white}
\setbeamercolor{block title}{bg=accentblue, fg=white}
\setbeamercolor{block body}{bg=accentblue!5, fg=black}
\setbeamercolor{item}{fg=accentblue}
\setbeamercolor{footline}{fg=gray}

\setbeamertemplate{navigation symbols}{}
\setbeamertemplate{footline}{%
  \leavevmode\hbox{%
    \begin{beamercolorbox}[wd=.5\paperwidth,ht=2.5ex,dp=1ex,left]{footline}%
      \hspace{1em}\scriptsize RetailSync -- Comparaison d'APIs
    \end{beamercolorbox}%
    \begin{beamercolorbox}[wd=.5\paperwidth,ht=2.5ex,dp=1ex,right]{footline}%
      \scriptsize\insertframenumber\,/\,\inserttotalframenumber\hspace{1em}
    \end{beamercolorbox}%
  }%
}

% -- Packages --
\usepackage[T1]{fontenc}
\usepackage[utf8]{inputenc}
\usepackage{lmodern}
\usepackage{graphicx}
\usepackage{booktabs}
\usepackage{array}
\usepackage{colortbl}
\usepackage{xcolor}
\usepackage{tikz}
\usepackage{hyperref}

\graphicspath{{../diagrams/}}

% -- Custom commands --
\newcommand{\soapbadge}{\colorbox{soapblue}{\textcolor{white}{\scriptsize\textbf{~SOAP~}}}}
\newcommand{\restbadge}{\colorbox{restgreen}{\textcolor{white}{\scriptsize\textbf{~REST~}}}}
\newcommand{\graphqlbadge}{\colorbox{graphqlpurple}{\textcolor{white}{\scriptsize\textbf{~GraphQL~}}}}
\newcommand{\grpcbadge}{\colorbox{grpcorange}{\textcolor{white}{\scriptsize\textbf{~gRPC~}}}}

\newcommand{\gained}[1]{\textcolor{restgreen}{\textbf{+}}~#1}
\newcommand{\sacrificed}[1]{\textcolor{red!70!black}{\textbf{--}}~#1}

% ======================================================================
% DOCUMENT
% ======================================================================
\title[RetailSync]{\textbf{RetailSync}}
\subtitle{Une première application pour comparer\\SOAP, REST, GraphQL et gRPC\\dans un système e-commerce}
\author{Module d'Architecture Logicielle}
\date{Février 2026}
\institute{}

\begin{document}

% ======================================================================
% TITLE
% ======================================================================
\begin{frame}[plain]
  \maketitle
\end{frame}

% ======================================================================
% AGENDA
% ======================================================================
\begin{frame}{Au programme}
  \begin{enumerate}
    \item \textbf{Le Contexte} -- Pourquoi utiliser 4 APIs différentes ? \hfill \textit{2 min}
    \item \textbf{L'Architecture} -- Comment tout se connecte \hfill \textit{3 min}
    \item \textbf{Plongée dans les 4 APIs} -- Démonstrations et avantages \hfill \textit{6 min}
    \item \textbf{Comparaison Pratique} -- Taille des messages et décisions \hfill \textit{5 min}
    \item \textbf{Conclusion} -- Ce qu'il faut retenir \hfill \textit{2 min}
  \end{enumerate}
  \vfill
  \centering
  \soapbadge\quad \restbadge\quad \graphqlbadge\quad \grpcbadge
\end{frame}

% ======================================================================
% SECTION 1: CONTEXT
% ======================================================================
\section{1. Le Contexte}

\begin{frame}{Le problème : 4 besoins très différents}
  Nous simulons une chaîne de magasins qui doit communiquer avec \textbf{quatre acteurs différents}. Chacun a ses propres contraintes :
  \vspace{0.3em}
  \begin{columns}[T]
    \begin{column}{0.48\textwidth}
      \begin{block}{\soapbadge~Les Usines (Achats B2B)}
        Vieux systèmes informatiques (SAP) très stricts. On a besoin d'une garantie totale sur le format des commandes, sans erreur.
      \end{block}
      \begin{block}{\graphqlbadge~Le(la) Gérant(e) (Tableau de bord)}
        A besoin de voir toutes les infos (stock, commandes) sur un seul écran, sans faire 10 requêtes différentes qui ralentissent le site.
      \end{block}
    \end{column}
    \begin{column}{0.48\textwidth}
      \begin{block}{\restbadge~Les Boutiques (Partenaires)}
        Veulent gérer leur stock facilement avec n'importe quel langage de programmation. Il faut faire très simple.
      \end{block}
      \begin{block}{\grpcbadge~Les Robots (Dans l'entrepôt)}
        Envoient leur position en continu (10 fois par seconde). Il faut que le message soit microscopique pour ne pas bloquer le réseau.
      \end{block}
    \end{column}
  \end{columns}
\end{frame}

\begin{frame}{Le code vient après}
  \begin{columns}[T]
    \begin{column}{0.55\textwidth}
      Pour chaque API de ce projet, nous avons suivi une règle stricte :
      \begin{enumerate}
        \item \textbf{Définir le ``Contrat''} -- On écrit d'abord un fichier qui décrit comment l'API va marcher.
        \item \textbf{Coder le serveur} -- Le code est créé à partir du contrat, jamais l'inverse.
        \item \textbf{Tester avec Postman} -- Pour vérifier que ça fonctionne bien !
      \end{enumerate}
    \end{column}
    \begin{column}{0.42\textwidth}
      \begin{table}
        \centering
        \scriptsize
        \begin{tabular}{ll}
          \toprule
          \textbf{Module} & \textbf{Le fichier Contrat} \\
          \midrule
          \textcolor{soapblue}{SOAP} & \texttt{PurchaseOrder.wsdl} \\
          \textcolor{restgreen}{REST} & \texttt{openapi.yaml} \\
          \textcolor{graphqlpurple}{GraphQL} & \texttt{schema.graphql} \\
          \textcolor{grpcorange}{gRPC} & \texttt{warehouse.proto} \\
          \bottomrule
        \end{tabular}
      \end{table}
      \vspace{0.5em}
      \begin{alertblock}{Idée à retenir}
        En séparant la règle (le contrat) du code, l'application est beaucoup plus facile à modifier et tester.
      \end{alertblock}
    \end{column}
  \end{columns}
\end{frame}

% ======================================================================
% SECTION 2: ARCHITECTURE
% ======================================================================
\section{2. L'Architecture}

\begin{frame}{Vue d'ensemble du Projet}
  \begin{center}
    \includegraphics[width=0.88\textwidth]{01_master_architecture.png}
  \end{center}
\end{frame}

% ======================================================================
% SECTION 3: DEEP DIVE
% ======================================================================
\section{3. Plongée dans les 4 APIs}

% -- SOAP --
\begin{frame}{\soapbadge~Module SOAP -- Commandes aux usines}
  \begin{columns}[T]
    \begin{column}{0.48\textwidth}
      \textbf{Format :} Lourd mais ultra sécurisé (XML)\\[0.8em]
      \textbf{Pourquoi on l'a choisi :}
      \begin{itemize}
        \item \textbf{Validation parfaite :} Impossible d'oublier un champ grâce au fichier WSDL qui vérifie tout.
        \item \textbf{Format d'entreprise :} C'est ce que les gros logiciels SAP/Oracle attendent.
      \end{itemize}
      \vspace{0.5em}
      \footnotesize
      \begin{tabular}{p{3.2cm}p{3.2cm}}
        \gained{Erreurs impossibles} & \sacrificed{XML dur à lire} \\
        \gained{Très sécurisé} & \sacrificed{Outils un peu vieux} \\
      \end{tabular}
    \end{column}
    \begin{column}{0.50\textwidth}
      \includegraphics[width=\textwidth]{02_soap_sequence.png}
    \end{column}
  \end{columns}
\end{frame}

% -- REST --
\begin{frame}{\restbadge~Module REST -- Gestion de l'inventaire}
  \begin{columns}[T]
    \begin{column}{0.48\textwidth}
      \textbf{Format :} Le grand classique du net (JSON)\\[0.8em]
      \textbf{Pourquoi on l'a choisi :}
      \begin{itemize}
        \item \textbf{Universel :} Tout le monde sait faire une requête GET ou PATCH.
        \item \textbf{Très facile à mettre en place :} Aucun fichier complexe à générer. Pas besoin d'outils compliqués.
      \end{itemize}
      \vspace{0.5em}
      \footnotesize
      \begin{tabular}{p{3.2cm}p{3.2cm}}
        \gained{Hyper simple} & \sacrificed{Pas de validation forte} \\
        \gained{Rapide à coder} & \sacrificed{Vieux design par requêtes séparées} \\
      \end{tabular}
    \end{column}
    \begin{column}{0.50\textwidth}
      \includegraphics[width=\textwidth]{03_rest_sequence.png}
    \end{column}
  \end{columns}
\end{frame}

% -- GraphQL --
\begin{frame}{\graphqlbadge~Module GraphQL -- Le Tableau de Bord}
  \begin{columns}[T]
    \begin{column}{0.48\textwidth}
      \textbf{Format :} Flexible (JSON)\\[0.8em]
      \textbf{Pourquoi on l'a choisi :}
      \begin{itemize}
        \item \textbf{Requête unique :} Parfait pour une interface web qui veut charger plein d'informations à la fois.
        \item \textbf{À la carte :} On ne télécharge que les informations qu'on veut afficher sur l'écran.
      \end{itemize}
      \vspace{0.5em}
      \footnotesize
      \begin{tabular}{p{3.2cm}p{3.2cm}}
        \gained{Une seule requête HTTP} & \sacrificed{Nouveau concept à apprendre} \\
        \gained{Économise internet} & \sacrificed{Difficile de mettre en cache} \\
      \end{tabular}
    \end{column}
    \begin{column}{0.50\textwidth}
      \includegraphics[width=\textwidth]{04_graphql_sequence.png}
    \end{column}
  \end{columns}
\end{frame}

% -- gRPC --
\begin{frame}{\grpcbadge~Module gRPC -- Les Robots de l'Entrepôt}
  \begin{columns}[T]
    \begin{column}{0.48\textwidth}
      \textbf{Format :} Binaire ultra léger (Protobuf)\\[0.8em]
      \textbf{Pourquoi on l'a choisi :}
      \begin{itemize}
        \item \textbf{Microscopique :} Un message fait 40 octets (contre 250 en REST). On gagne 6x de vitesse!
        \item \textbf{Temps réel :} Le serveur et le robot se parlent en continu sans refaire de nouvelle connexion (streaming bidirectionnel).
      \end{itemize}
      \vspace{0.5em}
      \footnotesize
      \begin{tabular}{p{3.2cm}p{3.2cm}}
        \gained{Ultra rapide / léger} & \sacrificed{Illisible sans outil} \\
        \gained{Streaming réel} & \sacrificed{Difficile pour le web} \\
      \end{tabular}
    \end{column}
    \begin{column}{0.50\textwidth}
      \includegraphics[width=\textwidth]{05_grpc_sequence.png}
    \end{column}
  \end{columns}
\end{frame}

% ======================================================================
% SECTION 4: CROSS-PARADIGM ANALYSIS
% ======================================================================
\section{4. Comparaison Pratique}

\begin{frame}{Pourquoi telle ou telle API ? (Arbre de décision)}
  \begin{center}
    \includegraphics[width=0.82\textwidth]{06_decision_tree.png}
  \end{center}
  \vspace{0.3em}
  \centering\small\textit{REST est le choix par défaut. On change seulement si on a un vrai problème technique à régler.}
\end{frame}

\begin{frame}{Comparatif de taille des messages}
  \begin{center}
    \includegraphics[width=0.85\textwidth]{08_payload_comparison.png}
  \end{center}
  \vspace{0.3em}
  \centering\small\textit{Pour envoyer la même information au système : 1 KB (SOAP) $\rightarrow$ 200 B (REST) $\rightarrow$ 50 B (gRPC).}
\end{frame}

\begin{frame}{GraphQL, le "Chef d'Orchestre"}
  \begin{center}
    \includegraphics[width=0.82\textwidth]{07_aggregator_pattern.png}
  \end{center}
  \vspace{0.3em}
  \centering\small\textit{Pour le développeur du site web, les différences entre SOAP, REST et gRPC sont invisibles. GraphQL unit tout !}
\end{frame}

% ======================================================================
% SECTION 5: COMPARISON MATRIX
% ======================================================================
\section{5. Conclusion}

\begin{frame}{Tableau de Synthèse}
  \begin{table}
    \centering
    \scriptsize
    \renewcommand{\arraystretch}{1.3}
    \begin{tabular}{l >{\color{soapblue}}c >{\color{restgreen}}c >{\color{graphqlpurple}}c >{\color{grpcorange}}c}
      \toprule
      \textbf{Critères} & \textbf{SOAP} & \textbf{REST} & \textbf{GraphQL} & \textbf{gRPC} \\
      \midrule
      Format & XML & JSON & JSON & Protobuf \\
      Connexion & HTTP/1.1 & HTTP/1.1 & HTTP/1.1 & HTTP/2 \\
      Sécurité format & Très stricte (WSDL) & Moyenne & Flexible & Stricte (.proto) \\
      Vitesse/Taille & Très lent (Lourd) & Basique & Economique & Ultra rapide (Léger) \\
      Streaming Direct & Non & Non & Partiel & Oui (Dans les 2 sens) \\
      Le Meilleur Pour & Tâches Critiques/B2B & Public et Simple & Sites Web (Affichage) & Robots et Temps Réel \\
      \bottomrule
    \end{tabular}
  \end{table}
  \vspace{0.5em}
  \centering\small Il n'y a pas d'API parfaite. Il n'y a que le \textbf{bon type d'API pour le bon besoin}.
\end{frame}

\begin{frame}{Que faut-il retenir de ce POC ?}
  \begin{enumerate}
    \item \textbf{Il n'y a pas de gagnant final.} Chaque API fait son travail parfaitement en fonction du besoin. L'avenir est de savoir les utiliser au bon endroit.

    \item \textbf{REST est notre valeur sûre.} C'est le choix facile et intelligent pour 80\% des projets classiques.

    \item \textbf{Le contrat avant le code.} Créer des règles (fichiers wsdl, proto, openapi) rend l'équipe de développement beaucoup plus autonome et réduit les erreurs.

    \item \textbf{GraphQL n'efface pas REST.} Il se met par-dessus, pour rendre l'interface web plus simple à utiliser sans tout casser derrière.

    \item \textbf{La vitesse gRPC est indispensable.} Pour des machines comme les robots travaillant 10 fois par seconde, SOAP ferait s'écrouler le réseau wifi de l'entrepôt, gRPC le laisse respirer.
  \end{enumerate}

  \vspace{0.5em}
\end{frame}

% ======================================================================
% THANK YOU
% ======================================================================
\begin{frame}[plain]
  \centering
  \vspace{2em}
  {\LARGE\textbf{Merci pour votre attention !}}\\[1.5em]
  {\large Avez-vous des questions ?}\\[2em]
  \soapbadge~\texttt{:8001}\quad
  \restbadge~\texttt{:8002}\quad
  \graphqlbadge~\texttt{:8003}\quad
  \grpcbadge~\texttt{:50051}\\[2em]
  {\footnotesize\texttt{github.com/AyaMor/omnichain-retail-mesh}}
\end{frame}

\end{document}
